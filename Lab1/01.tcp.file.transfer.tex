\documentclass{article}
\usepackage{graphicx}
\usepackage{hyperref}
\title{TCP File Transfer Using TCP Sockets}
\author{Dang Quang Minh - 23BI14304}

\begin{document}
\maketitle

\section{Introduction}
The goal of this practical work is to implement a simple 1-to-1 TCP file transfer system using C. The client sends a file to the server, which receives and stores it locally. The program is executed via command-line interface (CLI).

\section{System Architecture}
The system follows the client-server model. The server listens on a fixed port (9000). The client connects to the server and initiates a file transfer. The server accepts the connection, receives file and its content, and stores the received file.

\section{Protocol Design}
A custom lightweight protocol is used to transmit the file:
\begin{enumerate}
    \item Client sends filename length (4 bytes, integer)
    \item Client sends filename (string)
    \item Client sends file size (8 bytes, long long)
    \item Client sends file content in chunks
    \item Server writes data to disk
\end{enumerate}

TCP is a stream protocol, so metadata is necessary to help the server understand incoming data.

\section{Client Implementation}
The client code performs the following:
\begin{itemize}
    \item Opens a TCP socket
    \item Connects to the server
    \item Opens the file to send
    \item Sends metadata: filename length, filename, file size
    \item Sends file content
    \item Closes connection
\end{itemize}

\section{Server Implementation}
The server code:
\begin{itemize}
    \item Creates a TCP socket
    \item Binds to port 9000
    \item Listens for incoming connections
    \item Accepts a client connection
    \item Receives metadata and file content
    \item Saves the received file
    \item Closes socket
\end{itemize}

\section{Flow Diagram}
This figure below shows the communication flow between client and server.

\begin{figure}[h]
\centering
\includegraphics[width=0.8\textwidth]{diagram.png}
\caption{Client--Server Interaction Protocol}
\label{fig:diagram}
\end{figure}


\end{document}
